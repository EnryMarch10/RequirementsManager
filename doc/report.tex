\documentclass[12pt, a4paper]{report}
\usepackage[margin=1in]{geometry}
\usepackage[utf8]{inputenc}
\usepackage[T1]{fontenc}

\usepackage{times}

\usepackage{float}

\usepackage{url}
\usepackage{graphicx}
\usepackage{hyperref}

\usepackage{tabularray}

\usepackage[table, svgnames]{xcolor}
\usepackage[backend=biber, style=alphabetic, sorting=ynt]{biblatex}

% \addbibresource{report.bib}

\graphicspath{{img/}} % global configuration

\title{
    RequirementsManager
    % \begin{figure}[ht]
    % \centering
    % \includegraphics[width=\textwidth]{logo}
    % \end{figure}
}
\author{
    Enrico Marchionni\\
    \texttt{enrico.marchionni@studio.unibo.it}
}
\date{\today}

\begin{document}

\maketitle

\begin{abstract}

ReM\footnote{\emph{ReM}, Requirements Manager.} is an administrative software that makes you minimize time and maximize result in
the \emph{Software Engineering} process of collecting and creating requirements.

\end{abstract}

\tableofcontents

\addcontentsline{toc}{chapter}{Analysis}
\chapter*{Analysis}

It's requested to realize a database that allows a team to optimize Software's Analysis by managing Requirements creation.
Each one of them descends from a Request that could be created by a team member or by the project customers.

\addcontentsline{toc}{section}{Preliminary}
\section*{Preliminary}
\label{sec:preliminary}

\addcontentsline{toc}{subsection}{Interview}
\subsection*{Interview}
\label{subsec:interview}

The aim is to manage requirements during all phases of software production and revision storing away all final and in production
data relative to releases already published or yet to publish. Everything has to start from a \emph{Request}.
Once it has been created the relative \emph{Requirement/s} can be developed.

The system admin has to create users and the relative database linked to a software.
Requests can be created by a team member or a customer. Requirements can be developed by team members only. Requests and therefore
Requirements can obviously be added or even edited in later releases. It's not important to keep track of every update or change,
the important is to keep track of final data just before a release. The timeline can be added by a team member only and domain
conditions has to be respected.

Requirements have to be structured with a title, a type (functional, non functional), a version, a description, a body, one or
more files, progress data, a creation time and a last modified time. Moreover for a requirement it has to be known
the user that did the first and last modifies and an history of updates with times and users that did them. Furthermore is
important to consider that requirements could be arranged in a tree structure in the most complex cases.

Requests structure consists of a title, a type, a version, a description, a body, one or more files, a creation time and a last
modified time. Besides for a request it has to be known the user that did the first and last modifies and an history of
updates with times and users that did them.

A requirement has to be associated with one and only one request, a request can be associated with multiple requirements.

A request has to be approved by a team member before developing it into a requirement. After the approval a customer has to
accept the request approval, only then the relative requirement can be developed. When the approval is done the request mustn't
be able to be modified.

Users has to be saved with personal info such as username, name, surname, email, phone and optionally the company name.

The releases should also save a short description and a name that has to be an identifier between the timelines. It's important to
permit the creation of a new timeline only if every requirement was completed, or this mustn't be permitted. In addition every
request has to be approved, formalized into requirement and completed before the version creation.

Every data that was affected by the version cannot be overridden so their last modified time should be previous to the version
creation time.

% There has to be also the possibility to comment (and reply) requests and requirements so as to work in a more interactive and
% efficient way.

\addcontentsline{toc}{subsection}{Glossary}
\subsection*{Glossary}

A glossary of terms with description, synonyms and links used in \nameref{subsec:interview} is here explained in
\autoref{tab:glossary_terms}.

\begin{table}[H]
    \begin{tblr}{
        colspec={X[l]X[l]X[l]X[l]},
        width=\textwidth,
        row{odd}={gray!15},
        row{even}={white},
        row{1}={bg=gray!90,fg=white},
        colsep=4pt
      }
        \textbf{Term} & \textbf{Description} & \textbf{Synonyms} & \textbf{Links} \\
        Request & Customer request of what the system should do & & Requirement, release \\ %, comment \\
        \hline
        Requirement & Formal description of what the system should do & & Request, release \\ %, comment \\
        \hline
        & Requirements + Requests & Data \\
        \hline
        Editor & User that can add Requirements & Team member & \\
        \hline
        Guest & User that can only add Requests & Customer & \\
        \hline
        User & Editor + Guest & User & Editing, versioning \\ %, commenting \\
        \hline
        % Comment & A short text that comments something & Reply & Request, requirement \\
        Release & A timeline that confirms the fact that a software version was released  & Version, timeline & Data \\
        \hline
    \end{tblr}
    \caption{\label{tab:glossary_terms} Glossary of terms}
\end{table}

\addcontentsline{toc}{section}{Requirements}
\section*{Requirements}

This section aims to restructure requirements according to the \nameref{sec:preliminary} phase.

\addcontentsline{toc}{subsection}{Functional}
\subsection*{Functional}

\begin{itemize}
    \item General
    \begin{itemize}
        \item It is demanded to realize a database that manages requirements;
        \item Information that have to be stored are \textbf{Requests}, \textbf{Requirements}, \textbf{Users} and \textbf{Releases}; %and \textbf{Comments};
    \end{itemize}
    \item Users
    \begin{itemize}
        \item \textbf{Users} are structured with username, name, surname, email, phone and optionally the company name;
        \item \textbf{Users} are divided into two groups:
            \begin{itemize}
                \item \textbf{Editors} - they approve \emph{Requests}, develop relative \emph{Requirements} and decides \emph{Releases};
                    in some cases they could also create \emph{Requests};
                \item \textbf{Guests} - they create \emph{Requests} that once approved cannot be edited but will be developed into
                    \emph{Requirements};
            \end{itemize}
    \end{itemize}
    \item Requests
    \begin{itemize}
        \item \textbf{Requests} are structured with a title, a type, a version (positive integer), a description, a body, one or more
            files, a creation time and a last modified time;
    \end{itemize}
    \item Requirements
    \begin{itemize}
        \item \textbf{Requirements} are structured with a title, a type (functional, non functional), a version (positive integer),
            a description, a body, one or more files, progress data (a percentage), a creation time and a last modified time;
        \item \textbf{Requirements} could be arranged in a tree structure;
    \end{itemize}
    % \item Comments
    % \begin{itemize}
    %     \item \textbf{Comments} are structured with a description, the last modified time and the creation time;
    %     \item \textbf{Comments} are relative to \emph{Requests} and \emph{Requirements}; \emph{Editors} can comment both while
    %         \emph{Guests} can work only with \emph{Requests};
    % \end{itemize}
    \item Releases
    \begin{itemize}
        \item \textbf{Releases} are structured with a unique name (software version name), a short description and the creation time;
        \item \textbf{Releases} are timelines that can be created only if all \emph{Requests} had been approved and converted into
            already fully developed \emph{Requirements};
    \end{itemize}
    \item Conditions
    \begin{itemize}
        \item all \textbf{Requests} and \textbf{Requirements} relative to a \textbf{Release} have to be historicized;
        \item \textbf{Requests} and \textbf{Requirements} could be modified or even invalidated in later \textbf{Releases};
        \item a \textbf{Requirement} has to be associated with one and only one \textbf{Request}, a \textbf{Request} can be associated
            with multiple \textbf{Requirements}.
    \end{itemize}
\end{itemize}

\addcontentsline{toc}{subsection}{Non-Functional}
\subsection*{Non-Functional}

\begin{itemize}
    \item Data access has to be really fast;
    \item CRUD\footnote{\emph{CRUD}, Create, read, update, delete.} operations should be non blocking in final
        GUI\footnote{\emph{GUI}, Graphical user interface.};
\end{itemize}

\addcontentsline{toc}{section}{Actions}
\section*{Actions}

The mainly requested actions are:
\begin{enumerate}
    \item Subscribe a new user;
    \item View a request;
    \item Create a request;
    \item Update a request;
    \item Approve a request;
    \item View a requirement;
    \item Create a requirement;
    \item Update a requirement;
    \item Disable a requirement;
    \item Create a release;
    \item Show requirements and requests relative to a specific release;
    \item View single requirement history (every change for each release);
    \item Obtain an average of the progress time for completing a branch of the requirements tree;
\end{enumerate}

\addcontentsline{toc}{section}{Conceptual design}
\section*{Conceptual design}

In production I chose to use a mixed strategy to build the E/R diagram.

A first version of the E/R diagram explained in \ref{fig:ER_v1}.
This diagram represents the skeleton of the E/R diagram.
Here I made the "mistake" of mixing time concepts with static ones.
In fact Requests and Requirements are single entity sets but their historic of a Version (Release) is not well managed.

\begin{figure}[H]
\centering
\caption{Version 1}
\includegraphics[width=\textwidth]{E-R v1.png}
\label{fig:ER_v1}
\end{figure}

A second version of the E/R diagram explained in \ref{fig:ER_v2}.
Here a clean schema shows how Requests and Requirements should be traced in their historic versions.
Furthermore is important to watch that an historic version is always created when a new Release is published, independently from
Requests or Requirements changes.
This could lead to an inefficient use of storage...

\begin{figure}[H]
\centering
\caption{Version 2}
\includegraphics[width=\textwidth]{E-R v2.png}
\label{fig:ER_v2}
\end{figure}

A third version of the E/R diagram explained in \ref{fig:ER_v3}.
Here every attribute is present and shows which data are stored for each entity/relationship.

\begin{figure}[H]
\centering
\caption{Version 3}
\includegraphics[width=\textwidth]{E-R v3.png}
\label{fig:ER_v3}
\end{figure}

\addcontentsline{toc}{section}{Logical design}
\section*{Logical design}

\addcontentsline{toc}{subsection}{Table of volumes}
\subsection*{Table of volumes}

Data and expected volumes are described in \autoref{tab:volumes_data}.

\begin{table}[H]
    \begin{tblr}{
        colspec={X[l]cc},
        width=\textwidth,
        row{odd}={gray!15},
        row{even}={white},
        row{1}={bg=gray!90,fg=white},
        colsep=4pt
      }
        \textbf{Concept} & \textbf{Construct} & \textbf{Volume} \\
        USER & E & 100 \\
        \hline
        GUEST & E & 90 \\
        \hline
        EDITOR & E & 10 \\
        \hline
        REQUEST & E & 50 \\
        \hline
        HISTORIC\_REQUEST & E & 40 \\
        \hline
        REQUIREMENT & E & 70 \\
        \hline
        HISTORIC\_REQUIREMENT & E & 50 \\
        \hline
        RELEASE & E & 5 \\
        \hline
        DEVELOPMENT & R & 70 \\
        \hline
        HISTORIC\_DEVELOPMENT & R & 50 \\
        \hline
        KINSHIP & R & 5 \\
        \hline
        HISTORIC\_KINSHIP & R & 5 \\
        \hline
        CREATION\_RELEASE & R & 5 \\
        \hline
        RELEASE\_REQUESTS & R & 40 \\
        \hline
        RELEASE\_REQUIREMENTS & R & 50 \\
        \hline
        CREATION\_REQUEST & R & 50 \\
        \hline
        EDITING\_REQUEST & R & 50 \\
        \hline
        HISTORIC\_EDITING\_REQUEST & R & 40 \\
        \hline
        APPROVAL\_REQUEST & R & 45 \\
        \hline
        CREATION\_REQUIREMENT & R & 70 \\
        \hline
        EDITING\_REQUIREMENT & R & 70 \\
        \hline
        HISTORIC\_EDITING\_REQUIREMENT & R & 50 \\
        \hline
        ORIGINAL\_REQUEST & R & 40 \\
        \hline
        ORIGINAL\_REQUIREMENT & R & 50 \\
        \hline
    \end{tblr}
    \caption{\label{tab:volumes_data} Volumes of data}
\end{table}

\addcontentsline{toc}{subsection}{Operations and frequency}
\subsection*{Operations and frequency}

Operations and frequency are described in \autoref{tab:op_fr}.

\begin{table}[H]
    \begin{tblr}{
        colspec={cX[l]c},
        width=\textwidth,
        row{odd}={gray!15},
        row{even}={white},
        row{1}={bg=gray!90,fg=white},
        colsep=4pt
      }
        \textbf{Code} & \textbf{Operation} & \textbf{Frequency} \\
        1 & Subscribe a new user & 10 every month \\
        \hline
        2 & View a request & 200 every day \\
        \hline
        3 & Create a request & 2 every day \\
        \hline
        4 & Update a request & 5 every day \\
        \hline
        5 & Approve a request & 1 every day \\
        \hline
        6 & View a requirement & 200 every day \\
        \hline
        7 & Create a requirement & 2 every day \\
        \hline
        8 & Update a requirement & 50 every day \\
        \hline
        9 & Disable a requirement & 1 every month \\
        \hline
        10 & Create a release & 2 every year \\
        \hline
        11 & Show requirements and requests relative to a specific release & 50 every day \\
        \hline
        12 & View single requirement history (every change for each release) & 10 every day \\
        \hline
        13 & Obtain an average of the progress time for completing a branch of the requirements tree & 20 every day \\
        \hline
    \end{tblr}
    \caption{\label{tab:op_fr} Operations and frequency}
\end{table}

% \subsubsection*{Subscribe a new user}

% User subscription described in \autoref{tab:usr_sub}.

% \begin{table}[H]
%     \begin{tblr}{
%         colspec={X[l]X[l]X[l]X[l]},
%         width=\textwidth,
%         row{odd}={gray!15},
%         row{even}={white},
%         row{1}={bg=gray!90,fg=white},
%         colsep=4pt
%       }
%         \textbf{Concept} & \textbf{Construct} & \textbf{Access} & \textbf{Type} \\
%         User & E & 1 & W \\
%     \end{tblr}
%     \caption{\label{tab:usr_sub} Subscribe a new user}
% \end{table}

% \subsubsection*{View a request}

% Request visualization described in \autoref{tab:vw_req}.

% \begin{table}[H]
%     \begin{tblr}{
%         colspec={X[l]X[l]X[l]X[l]},
%         width=\textwidth,
%         row{odd}={gray!15},
%         row{even}={white},
%         row{1}={bg=gray!90,fg=white},
%         colsep=4pt
%       }
%         \textbf{Concept} & \textbf{Construct} & \textbf{Access} & \textbf{Type} \\
%         Request & E & 1 & L \\
%     \end{tblr}
%     \caption{\label{tab:vw_req} Subscribe a new user}
% \end{table}

% \appendix

% \printbibliography

\end{document}